
\documentclass{article}
\usepackage[UTF8]{ctex}
\usepackage{geometry}
\geometry{a4paper, margin=1in}

\title{C++中vector的介绍}
\author{}
\date{}

\begin{document}

\maketitle

\section{简介}
在C++中,\texttt{vector} 是标准模板库(STL)中最常用的顺序容器之一。它可以理解为一个动态数组,能够根据需要自动调整大小,并且提供了快速的随机访问和高效的尾部插入操作。

\section{vector的特点}
\begin{itemize}
    \item \textbf{动态大小}:\texttt{vector} 可以自动扩展或收缩,不需要提前指定大小。
    \item \textbf{随机访问}:支持常数时间的随机访问,类似于普通数组。
    \item \textbf{自动管理内存}:\texttt{vector} 自动管理其内存分配,用户不需要手动分配和释放内存。
    \item \textbf{高效的尾部操作}:在末尾插入、删除元素的时间复杂度是常数 \(O(1)\),但在中间或开头插入、删除元素的操作时间复杂度为线性 \(O(n)\)。
\end{itemize}

\section{常见操作}
\texttt{vector} 提供了丰富的操作函数,常见操作包括:

\subsection{初始化与定义}
\begin{itemize}
    \item 默认构造函数:\texttt{vector<int> v;}
    \item 初始化指定大小:\texttt{vector<int> v(5);} 初始化大小为5,所有元素默认初始化为0。
    \item 初始化大小和默认值:\texttt{vector<int> v(5, 10);} 初始化大小为5,每个元素的值都是10。
    \item 列表初始化:\texttt{vector<int> v = \{1, 2, 3, 4, 5\};}
\end{itemize}

\subsection{添加元素}
\begin{itemize}
    \item \texttt{push\_back}:在末尾添加元素。
    \item \texttt{insert}:在指定位置插入元素。
\end{itemize}

\subsection{删除元素}
\begin{itemize}
    \item \texttt{pop\_back}:删除最后一个元素。
    \item \texttt{erase}:删除指定位置的元素或范围。
\end{itemize}

\subsection{访问元素}
\begin{itemize}
    \item 使用索引:\texttt{v[i]}
    \item 使用 \texttt{at} 函数:\texttt{v.at(i)},这种方式会进行边界检查。
    \item 获取第一个和最后一个元素:\texttt{front()} 和 \texttt{back()}。
\end{itemize}

\subsection{遍历元素}
\begin{itemize}
    \item 使用迭代器遍历:\texttt{begin()} 和 \texttt{end()}。
    \item 使用范围 \texttt{for} 循环。
\end{itemize}

\subsection{容量管理}
\begin{itemize}
    \item \texttt{size()}:当前元素个数。
    \item \texttt{capacity()}:\texttt{vector} 容器的当前容量。
    \item \texttt{resize()}:调整容器大小。
    \item \texttt{reserve()}:调整容量,预留空间。
\end{itemize}

\section{示例代码}

下面展示一些使用 \texttt{vector} 的代码示例。

\subsection{基本操作示例}

\begin{verbatim}
#include <iostream>
#include <vector>
using namespace std;

int main() {
    vector<int> v;

    // 添加元素
    v.push_back(10);
    v.push_back(20);
    v.push_back(30);

    // 输出元素
    cout << "Vector elements: ";
    for (int i = 0; i < v.size(); i++) {
        cout << v[i] << " ";
    }
    cout << endl;

    // 使用 at 函数访问
    cout << "Element at index 1: " << v.at(1) << endl;

    // 获取第一个和最后一个元素
    cout << "First element: " << v.front() << endl;
    cout << "Last element: " << v.back() << endl;

    // 删除最后一个元素
    v.pop_back();
    cout << "After pop_back, size: " << v.size() << endl;

    return 0;
}
\end{verbatim}

\subsection{使用迭代器遍历 \texttt{vector}}

\begin{verbatim}
#include <iostream>
#include <vector>
using namespace std;

int main() {
    vector<int> v = {1, 2, 3, 4, 5};

    // 使用迭代器遍历
    cout << "Using iterator to print elements: ";
    for (vector<int>::iterator it = v.begin(); it != v.end(); ++it) {
        cout << *it << " ";
    }
    cout << endl;

    return 0;
}
\end{verbatim}

\subsection{插入和删除元素}

\begin{verbatim}
#include <iostream>
#include <vector>
using namespace std;

int main() {
    vector<int> v = {1, 2, 3, 4, 5};

    // 在指定位置插入元素
    v.insert(v.begin() + 2, 10);  // 在索引 2 处插入 10

    // 删除指定位置的元素
    v.erase(v.begin() + 4);  // 删除索引 4 处的元素

    // 输出修改后的 vector
    cout << "Modified vector: ";
    for (int num : v) {
        cout << num << " ";
    }
    cout << endl;

    return 0;
}
\end{verbatim}

\section{总结}
\texttt{vector} 是一个功能强大且灵活的动态数组,它提供了方便的内存管理和常见的操作函数。在使用 \texttt{vector} 时,不仅可以享受类似数组的高效随机访问,还可以方便地动态调整其大小,非常适合处理需要频繁插入、删除和遍历的场景。

\end{document}
