\documentclass[12pt]{article}
\usepackage[utf8]{inputenc}
\usepackage[margin=1in]{geometry}
\usepackage[english]{babel}
\usepackage[UTF8]{ctex}
\title{哈希表的重新哈希与C++标准库中的哈希表实现}
\author{}
\date{}

\begin{document}

\maketitle

\section{哈希表的重新哈希(Rehashing)}
如果哈希表变得太满,操作的运行时间将开始变得太长,插入可能会因为二次探测开放寻址哈希表而失败。这可能会发生在插入和删除太多混合的情况下。一个解决方案是构建另一个大约是原来两倍大的表(以及一个相关的新哈希函数),然后扫描整个原始哈希表,为每个(未被删除的)元素计算新的哈希值,并将其插入到新表中。

作为一个例子,假设元素13、15、24和6被插入到一个使用线性探测的哈希表中。随着时间的推移,表变得太满,我们需要进行重新哈希。我们将构建一个更大的新表,并将所有元素重新哈希到这个新表中。这个过程需要扫描整个原始哈希表,对于每个元素,我们使用新的哈希函数计算其在新表中的位置,并将元素插入到那里。

这个过程可以确保哈希表在插入和删除操作后仍然保持较高的性能。重新哈希是一个昂贵的操作,因为它需要遍历整个哈希表,但对于保持哈希表性能来说这是必要的。在实际应用中,我们通常会监控哈希表的负载因子,并在它超过某个阈值时进行重新哈希,以避免性能下降。

\section{C++标准库中的哈希表实现}
在C++11标准中引入了\texttt{unordered\_map}和\texttt{unordered\_set}这两个类模板,它们分别提供了基于哈希表的map和set的实现。这些容器提供了快速的插入、删除和搜索操作,这些操作的平均时间复杂度为O(1)。这些容器的实现通常使用哈希表,但具体的实现细节对用户是透明的。

\subsection{unordered\_map和unordered\_set}
\texttt{unordered\_map}和\texttt{unordered\_set}容器使用哈希表来存储元素,但它们并不保证元素的顺序。这些容器的关键在于哈希函数的设计和冲突解决策略。在C++标准库中,这些容器的哈希函数通常是可配置的,用户可以提供自定义的哈希函数来满足特定的需求。

\subsection{性能考虑}
虽然\texttt{unordered\_map}和\texttt{unordered\_set}提供了快速的操作,但是它们在最坏情况下的性能可能会下降到O(n),这通常发生在哈希函数不能均匀分布元素时。因此,选择合适的哈希函数对于这些容器的性能至关重要。

\subsection{使用示例}
使用\texttt{unordered\_map}和\texttt{unordered\_set}非常简单,用户可以像使用其他标准库容器一样使用它们。例如,\texttt{unordered\_map}可以用于存储键值对,而\texttt{unordered\_set}可以用于存储唯一的元素。这些容器提供了插入、删除和查找元素的操作,以及一些其他的实用功能,如size()和empty()。

\subsection{注意事项}
使用\texttt{unordered\_map}和\texttt{unordered\_set}时需要注意的是,由于它们基于哈希表,所以对元素的哈希值的计算可能会比较昂贵,特别是对于复杂的数据类型。此外,这些容器不保证元素的顺序,如果需要有序的元素,应该使用map和set。

总的来说,\texttt{unordered\_map}和\texttt{unordered\_set}为C++提供了一种高效的键值存储和集合操作的方式,但用户需要注意选择合适的哈希函数,并考虑到它们在最坏情况下的性能。

\end{document}