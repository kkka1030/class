\documentclass[a4paper,12pt]{article}
\usepackage{amsmath}
\usepackage{amssymb}
\usepackage{geometry}
\usepackage{listings}
\usepackage{xcolor}
\usepackage[UTF8]{ctex}
\geometry{a4paper, margin=1in}

\lstset{
  language=C++,
  basicstyle=\ttfamily\small,
  keywordstyle=\bfseries\color{blue},
  commentstyle=\itshape\color{green!50!black},
  stringstyle=\color{red},
  numbers=left,
  numberstyle=\tiny,
  stepnumber=1,
  numbersep=5pt,
  showstringspaces=false,
  breaklines=true,
  frame=single,
  rulecolor=\color{black},
  tabsize=2
}

\title{5.4: 不使用链表的哈希表}
\author{}
\date{}

\begin{document}

\maketitle

\section*{概述}
与分离链表哈希表相比,分离链表哈希表的缺点是使用了链表。这可能会因为分配新单元所需的时间而稍微降低算法的速度,并且实际上需要实现第二种数据结构。不使用链表来解决冲突的替代方法是尝试替代的单元,直到找到一个空单元。更正式地说,单元 h0(x), h1(x), h2(x), ... 被连续尝试,其中 h(x) = (hash(x) + f(i)) mod TableSize,f(0) = 0。函数 f 是冲突解决策略。由于所有数据都进入表中,因此需要一个比分离链表哈希表更大的表。通常,对于不使用分离链表的哈希表,负载因子应该低于 λ = 0.5。我们称这样的表为探测哈希表。我们现在来看三种常见的冲突解决策略。

---

\section{5.4.1 线性探测法(Linear Probing)}
线性探测法从发生冲突的槽位开始,逐个检查下一个槽位,直到找到空槽。

\subsection*{算法步骤}
\begin{itemize}
    \item \textbf{插入}:从初始哈希值开始,检查每个槽位,找到空槽插入。
    \item \textbf{查找}:从初始槽位依次检查,直到找到目标或遇到空槽。
    \item \textbf{删除}:标记为“已删除”(lazy deletion),避免影响其他元素。
\end{itemize}

\subsection*{优缺点}
\begin{itemize}
    \item \textbf{优点}:实现简单,元素存储在同一数组中。
    \item \textbf{缺点}:容易产生“聚集现象”,降低性能。
\end{itemize}

---

\section{5.4.2 二次探测法(Quadratic Probing)}
二次探测法通过非线性跳跃减少聚集现象。

\subsection*{算法公式}
\[
f(i) = i^2 \bmod M
\]
其中,\( i \) 是探测次数,\( M \) 是哈希表的大小。

\subsection*{优缺点}
\begin{itemize}
    \item \textbf{优点}:减少聚集现象。
    \item \textbf{缺点}:需要仔细选择 \( M \),否则可能无法插入。
\end{itemize}

\subsection*{定理5.1}
定理5.1 如果使用二次探测,并且表大小是质数,那么如果表至少半空,总是可以插入新元素。(证明详见课本)


\section{5.4.3 双重哈希(Double Hashing)}
双重哈希使用两个独立的哈希函数,进一步优化冲突处理。

\subsection*{算法公式}
\[
f(i) = i·h_2(x)
\]

\subsection*{举例}
如果选择不当,hash2(x) 可能会导致灾难性的后果。例如,如果插入了99,那么显而易见的选择$ hash2(x) = x mod 9 $就不会有帮助。因此,该函数必须永远不会评估为零。同样重要的是要确保所有的单元格都可以被探测到(在下面的例子中这是不可能的,因为表大小不是质数)。一个函数如 $hash2(x) = R - (x mod R)$,其中 R 是一个小于 TableSize 的质数,将会工作得很好。

\subsection*{优缺点}
\begin{itemize}
    \item \textbf{优点}:分布更均匀,进一步减少聚集。
    \item \textbf{缺点}:实现复杂,需要设计两个独立的哈希函数。
\end{itemize}

---

\section{方法对比}
\begin{table}[h!]
\centering
\begin{tabular}{|l|l|l|l|}
\hline
\textbf{方法}      & \textbf{探测公式}                              & \textbf{优点}           & \textbf{缺点}           \\ \hline
线性探测法         & \( h_i(k) = (h(k) + i) \bmod M \)            & 实现简单               & 易出现聚集现象         \\ \hline
二次探测法         & \( h_i(k) = (h(k) + i^2) \bmod M \)          & 减少聚集现象           & 插入失败可能性更高     \\ \hline
双重哈希           & \( h_i(k) = (h_1(k) + i \cdot h_2(k)) \bmod M \) & 分布更均匀             & 实现复杂,需要两个哈希 \\ \hline
\end{tabular}
\caption{哈希表冲突处理方法对比}
\end{table}

---

\section{示例代码:线性探测法}
以下代码实现了线性探测法的插入、查找和删除功能:
\begin{lstlisting}[caption=基于线性探测法的哈希表实现]
#include <iostream>
#include <vector>
#include <string>
using namespace std;

class HashTable {
private:
    vector<string> table;
    int size;
    string DELETED = "__DELETED__";

public:
    HashTable(int size) : size(size) {
        table.resize(size, "");
    }

    int hashFunction(string key) {
        int hash = 0;
        for (char ch : key) {
            hash = (hash * 31 + ch) % size;
        }
        return hash;
    }

    void insert(string key) {
        int index = hashFunction(key);
        while (table[index] != "" && table[index] != DELETED) {
            index = (index + 1) % size; 
        }
        table[index] = key;
    }

    bool search(string key) {
        int index = hashFunction(key);
        int start = index;
        while (table[index] != "") {
            if (table[index] == key) return true;
            index = (index + 1) % size;
            if (index == start) break; 
        }
        return false;
    }

    void remove(string key) {
        int index = hashFunction(key);
        int start = index;
        while (table[index] != "") {
            if (table[index] == key) {
                table[index] = DELETED;
                return;
            }
            index = (index + 1) % size;
            if (index == start) break;
        }
    }

    void display() {
        for (int i = 0; i < size; ++i) {
            cout << i << ": " << table[i] << endl;
        }
    }
};

int main() {
    HashTable hashTable(7);

    hashTable.insert("Alice");
    hashTable.insert("Bob");
    hashTable.insert("Charlie");

    hashTable.display();

    cout << "Search Bob: " << hashTable.search("Bob") << endl;

    hashTable.remove("Bob");
    cout << "After removing Bob:" << endl;
    hashTable.display();

    return 0;
}
\end{lstlisting}

---

\end{document}
