\documentclass{article}
\usepackage{amsmath}
\usepackage{listings}
\usepackage{graphicx}
\usepackage{geometry}
\usepackage[UTF8]{ctex}
\geometry{a4paper, margin=1in}

\title{BST remove 函数修改}
\author{}
\date{}

\begin{document}

\maketitle

\section{简介}
本报告介绍了二叉搜索树项目中 \texttt{remove} 函数的修改实现,并包含详细的测试和分析。主要目标是提升删除操作的效率,避免递归调用,并防止节点内容复制。在删除特定节点时考虑了各种情况,确保了树结构的完整性。

\section{\texttt{remove} 函数的实现}
在此项目中,\texttt{remove} 函数进行了如下优化:

1. 非递归实现:原始的 \texttt{remove} 函数通过递归方式寻找并删除目标节点。修改后的 \texttt{remove} 采用非递归实现,避免了递归调用带来的栈开销,提升了效率。
2. 避免节点内容复制:在删除具有两个子节点的节点时,我们采用一个 \texttt{detachMin} 辅助函数来分离右子树中的最小节点,以替换被删除节点。通过指针交换的方式直接替换,而不是复制内容,这样可以有效地减少内存开销。
3. 节点分离函数 \texttt{detachMin}:用于从给定子树中分离出最小节点。它将最小节点从树中删除并返回,用于替换被删除节点,实现了树结构的维护。

\subsection{修改后的 \texttt{remove} 函数代码}
以下是修改后的 \texttt{remove} 函数代码:

\begin{lstlisting}[language=C++]
    BinaryNode* detachMin(BinaryNode*& t) 
    {
    if (t == nullptr)
    {
        return nullptr;
    }
    BinaryNode* parent = nullptr;
    BinaryNode* current = t;
    
    while (current->left != nullptr) 
    {
        parent = current;
        current = current->left;
    }

    if (parent != nullptr) 
    {
        parent->left = current->right; 
    }
    else 
    {
        t = current->right; 
    }
    
    current->left = nullptr;
    current->right = nullptr;
    return current; 
    }

    void remove( const Comparable & x, BinaryNode * & t )
    {
        BinaryNode* parent = nullptr;
        BinaryNode* current = t;
        while (current != nullptr && current->element != x) 
        {
            parent = current;
            if (x < current->element) 
            {
                current = current->left;
            } 
            else 
            {
                current = current->right;
            }
        }
        if( current == nullptr )
        {    
            return;   // Item not found; do nothing
        }
        if( current->left != nullptr && current->right != nullptr ) // Two children
        {
            BinaryNode* m = detachMin( current->right );
            m->left = current->left;
            m->right = current->right;
            if (parent == nullptr)
            {
                t = m; 
            } 
            else if (parent->left == current) 
            {
                parent->left = m;
            } 
            else 
            {
                parent->right = m;
            }
        delete current;
        }
        else
        {
            if (parent == nullptr)
            {
                t = ( current->left != nullptr ) ? current->left : current->right;
            } 
            else if (parent->left == current)
            {
                parent->left = ( current->left != nullptr ) ? current->left : current->right;                
            }
            else 
            {
                parent->right = ( current->left != nullptr ) ? current->left : current->right;
            }
            
            delete current;
        }
    }
\end{lstlisting}

\section{测试结果与分析}
我们设计了几个测试用例,来验证 \texttt{remove} 函数在不同情况下的表现。测试结果如下:

1. 删除叶子节点:测试删除叶子节点(如节点3)。删除后树的结构保持正确,且该节点不可再被查找。
2. 删除仅有一个子节点的节点:测试删除仅有一个子节点的节点(如节点5),并验证其唯一子节点正确连接到父节点。
3. 删除有两个子节点的节点:测试删除有两个子节点的节点(如根节点10)。删除后,右子树的最小节点(节点12)被用来替代原节点,且树的结构保持了二叉搜索树的特性。
4. 删除不存在的节点:测试删除一个不存在的节点(如100),确保删除操作安全退出且不会影响树的结构。

\subsection{测试输出示例}
以下是测试输出的部分示例:

\begin{verbatim}
插入元素: 10, 5, 15, 3, 7, 12, 18
当前树的内容:
3 5 7 10 12 15 18

测试删除叶子节点
删除元素 3
删除后树的内容:
5 7 10 12 15 18
检查删除后查找元素 3: 未找到

测试删除只有一个子节点的节点
删除元素 5
删除后树的内容:
7 10 12 15 18
检查删除后查找元素 5: 未找到

测试删除有两个子节点的节点
删除元素 10
删除后树的内容:
7 12 15 18
检查删除后查找元素 10: 未找到
\end{verbatim}

\subsection{结果分析}
测试结果表明,修改后的 \texttt{remove} 函数在所有情况下均表现正确。以下是各测试用例的结果分析:

\begin{itemize}
    \item 叶子节点删除:删除后,树的结构未受影响,查找该节点时返回未找到。
    \item 仅有一个子节点的节点删除:子节点正确接到父节点,删除操作后符合预期。
    \item 有两个子节点的节点删除:右子树的最小节点替代被删除节点,实现了无内容复制的结构调整。
    \item 不存在节点的删除:删除操作安全退出,没有影响树的结构。
\end{itemize}

\section{总结}
本次修改优化了 \texttt{remove} 函数,使其在删除节点时避免递归调用和内容复制,提升了效率。测试结果表明,\texttt{remove} 函数在所有情况中均表现正确,树的结构在多种操作下均保持了二叉搜索树的性质,验证了修改的有效性。

\end{document}
