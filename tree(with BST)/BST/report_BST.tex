\documentclass[a4paper,12pt]{report}
\usepackage[UTF8]{ctex}
\usepackage{amsmath} % 数学符号
\usepackage{geometry} % 页面设置
\geometry{left=2.5cm,right=2.5cm,top=3cm,bottom=3cm}
\usepackage{graphicx} % 插入图片

\title{二叉搜索树基础报告}
\author{邵奕涵}
\date{}

\begin{document}

\maketitle
\tableofcontents % 自动生成目录

\chapter{简介}
二叉搜索树(Binary Search Tree, 简称 BST)是一种特殊的二叉树。它具有以下性质:
\begin{itemize}
    \item 对于每个节点,左子树的所有节点的值都小于该节点的值;
    \item 对于每个节点,右子树的所有节点的值都大于该节点的值;
    \item 左右子树也是二叉搜索树。
\end{itemize}
二叉搜索树广泛应用于数据查找、插入、删除等场景中,因其效率高且实现简单,成为计算机科学中的重要数据结构。

\chapter{二叉搜索树的操作}
\section{查找(Search)}
在二叉搜索树中查找一个值时,步骤如下:
\begin{itemize}
    \item 从根节点开始,与查找值比较;
    \item 如果目标值小于当前节点的值,则进入左子树;
    \item 如果目标值大于当前节点的值,则进入右子树;
    \item 如果找到相等的节点,则查找成功,否则继续直到找到或遍历完树。
\end{itemize}

\section{插入(Insert)}
在二叉搜索树中插入一个新的节点时,按以下步骤进行:
\begin{itemize}
    \item 从根节点开始,按照查找的方式找到合适的插入位置;
    \item 当找到空位时,将新节点插入,使得树仍然保持二叉搜索树的性质。
\end{itemize}

\section{删除(Delete)}
删除二叉搜索树中的节点有三种情况:
\begin{itemize}
    \item 删除的节点是\textbf{叶子节点},则可以直接删除。
    \item 删除的节点有一个子节点,则让该节点的父节点指向其唯一的子节点。
    \item 删除的节点有两个子节点,找到该节点右子树中的最小节点(后继节点),用后继节点的值替换该节点,然后删除后继节点。
\end{itemize}

\chapter{二叉搜索树的遍历}
遍历二叉搜索树时,常见的方式有:
\begin{itemize}
    \item \textbf{前序遍历}:先访问根节点,再访问左子树,最后访问右子树。
    \item \textbf{中序遍历}:先访问左子树,再访问根节点,最后访问右子树。对二叉搜索树进行中序遍历可以得到有序的结果。
    \item \textbf{后序遍历}:先访问左子树,再访问右子树,最后访问根节点。
    \item \textbf{层序遍历}:按照从上到下、从左到右的顺序逐层访问树中的每个节点。
\end{itemize}

\chapter{时间复杂度}
在理想情况下,二叉搜索树的深度约为 $\log n$,因此查找、插入、删除操作的平均时间复杂度为 $O(\log n)$。但在最坏情况下,二叉搜索树会退化成链表,此时操作的时间复杂度为 $O(n)$。

\chapter{二叉搜索树的应用}
\section{数据存储与检索}
由于二叉搜索树支持快速的查找、插入和删除操作,它常用于实现数据库索引、字典等。

\section{排序}
通过对二叉搜索树进行中序遍历,可以很方便地得到一个有序的序列,这使它在排序算法中具有应用价值。

\section{平衡二叉树}
为了避免二叉搜索树退化为链表,出现了诸如 AVL 树和红黑树等平衡二叉树,这些改进的树结构在保持操作效率的同时避免了最坏情况的发生。

\chapter{改进与平衡树}
为了解决二叉搜索树可能不平衡的问题,出现了以下几种平衡树:
\begin{itemize}
    \item \textbf{AVL树}:严格平衡的二叉搜索树,左右子树高度差不超过1。
    \item \textbf{红黑树}:一种近似平衡的二叉树,应用广泛,特别是在C++ STL中的集合和映射中。
\end{itemize}

\chapter{结论}
二叉搜索树是一种简单而高效的数据结构,具有广泛的应用。尽管在极端情况下它会退化成链表,但通过引入平衡二叉树的概念,可以避免性能退化,从而确保其在查找、插入和删除操作中的高效性。

\end{document}
