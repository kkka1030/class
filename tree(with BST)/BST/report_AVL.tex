\documentclass{article}

\usepackage{amsmath}
\usepackage[UTF8]{ctex}

\title{\vspace{-5.5cm}AVL 树删除操作实现报告}
\author{}
\date{}

\begin{document}

\maketitle
\vspace{-6em}

\section*{1. 基础删除逻辑}

AVL 树中的删除操作基本上与标准二叉搜索树的删除类似,主要包括以下几种情况:
\begin{itemize}
    \item \textbf{无子节点}:直接删除该节点。
    \item \textbf{一个子节点}:将该节点替换为其唯一的子节点,然后删除。
    \item \textbf{两个子节点}:找到\textbf{右子树的最小节点},将该节点的值替换到当前节点,然后递归删除右子树中的最小节点。
\end{itemize}



\section*{2. 维持平衡性}

在删除节点后,树的平衡性可能会受到影响,因此我们在删除操作中引入了平衡调整:
\begin{itemize}
    \item \textbf{\texttt{height}}:给节点的结构里增加了height
    \item \textbf{\texttt{updateHeight}}:更新节点的高度信息,确保平衡检查的准确性。
    \item \textbf{\texttt{balance}}:检查每个节点的左右子树高度差是否超过允许的最大不平衡值(1)。如果超过,则根据结构进行旋转操作。
    \begin{itemize}
        \item \textbf{\texttt{rotateWithLeftChild}}:左单旋转
        \item \textbf{\texttt{rotateWithRightChild}}:左单旋转
        \item \textbf{\texttt{doubleWithLeftChild}}:左右双旋转
        \item \textbf{\texttt{doubleWithRightChild}}:右左双旋转
    \end{itemize}
\end{itemize}

\section*{3. 平衡调整流程}

\begin{enumerate}
    \item 删除操作完成后,逐层回溯调用 \texttt{updateHeight},更新节点的高度。
    \item 调用 \texttt{balance} 检查当前节点的平衡性。如果左右子树的高度差超过允许的不平衡值,\texttt{balance} 函数会根据不平衡的类型选择适当的旋转操作以恢复平衡。
\end{enumerate}

\section*{4.特别的}
为了避免不必要的数据拷贝操作,我们使用了智能指针 \texttt{std::unique\_ptr} 来管理节点中的数据。在替换节点时,通过交换指针来减少数据拷贝,从而提高性能。


\end{document}
