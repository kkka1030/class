\documentclass[UTF8]{ctexart}
\usepackage{geometry, CJKutf8}
\geometry{margin=1.5cm, vmargin={0pt,1cm}}
\setlength{\topmargin}{-1cm}
\setlength{\paperheight}{29.7cm}
\setlength{\textheight}{25.3cm}

% useful packages.
\usepackage{amsfonts}
\usepackage{amsmath}
\usepackage{amssymb}
\usepackage{amsthm}
\usepackage{enumerate}
\usepackage{graphicx}
\usepackage{multicol}
\usepackage{fancyhdr}
\usepackage{layout}
\usepackage{listings}
\usepackage{float, caption}

\lstset{
    basicstyle=\ttfamily, basewidth=0.5em
}

% some common command
\newcommand{\dif}{\mathrm{d}}
\newcommand{\avg}[1]{\left\langle #1 \right\rangle}
\newcommand{\difFrac}[2]{\frac{\dif #1}{\dif #2}}
\newcommand{\pdfFrac}[2]{\frac{\partial #1}{\partial #2}}
\newcommand{\OFL}{\mathrm{OFL}}
\newcommand{\UFL}{\mathrm{UFL}}
\newcommand{\fl}{\mathrm{fl}}
\newcommand{\op}{\odot}
\newcommand{\Eabs}{E_{\mathrm{abs}}}
\newcommand{\Erel}{E_{\mathrm{rel}}}

\begin{document}

\pagestyle{fancy}
\fancyhead{}
\lhead{邵奕涵, 3230100001}
\chead{数据结构与算法第四次作业}
\rhead{Oct.19th, 2024}

\section{测试程序的设计思路}

在设计测试程序时,我首先确保每个功能点都得到了验证,包括基本的插入和删除操作、迭代器操作(前置递增和后置递增)、范围删除(`erase`)、以及链表的清空操作。

我测试了如下功能:
\begin{itemize}
    \item 我创建了一个 `List<int>` 对象,也测试了初始化列表构造函数。
    \item 测试了使用 `push\_back()` 和 `push\_front()` 函数插入多个元素。
    \item 测试了 `front()` 和 `back()` 函数,以验证它们是否正确返回链表的第一个和最后一个元素。
    \item 测试了 pop\_back()和pop\_front()删除头尾元素
    \item 使用前置递增 `++iterator` 和后置递增 `iterator++` 分别遍历链表,确保迭代器的行为正确。
    \item 测试了erase单个元素和erase(iterator from, iterator to) 删除指定范围内的元素,检查链表在删除后的状态。
    \item 测试了insert函数。
    \item 对 `clear()` 函数进行测试,清空链表,并验证链表的大小和状态。
    \item 测试了拷贝构造和移动构造,验证链表在深拷贝和资源转移后的行为是否正确。
    \item 测试了赋值运算符和移动赋值运算符
    \item 测试了begin()和end()
    \item 测试了empty()和size()
    
\end{itemize}

我的测试充分覆盖了链表的所有主要操作,包括插入、删除、遍历、赋值、拷贝和移动语义等。

\section{测试的结果}

测试的终端输出结果如下:
\begin{verbatim}
测试: 默认构造函数成功
测试push_back和push_front: 插入 5, 10, 15,20,25 到链表
5 10 15 20 25
测试: front() 返回 5
测试: back() 返回 25
测试pop_back()和pop_front():删除头尾元素,此时链表应剩余 10 15 20
10 15 20
测试前置递增 ++iterator
当前元素为: 15
测试后置递增 iterator++
当前元素为: 10
测试初始化列表构造函数
1 2 3 4 5
测试拷贝构造函数
10 15 20
测试移动构造函数
10 15 20
测试insert: 在链表头部插入 100, 尾部插入 200
100 10 15 20 200
测试erase: 删除头部元素
10 15 20 200
测试erase(from, to)
10 200
测试赋值运算符
10 200
测试移动赋值运算符
10 200
测试clear() 函数清空链表
测试empty()和size()成功
测试空链表中begin()和end()成功
\end{verbatim}

从测试结果中可以看到:
\begin{itemize}
    \item `push\_back()` 和 `push\_front()` 成功将元素插入到链表中,输出结果与预期一致。
    \item `front()` 函数返回了链表的第一个元素,`back()` 返回了最后一个元素,测试成功。
    \item `pop\_back()` 和 `pop\_front()` 成功删除了链表的头部和尾部元素,链表中的剩余元素符合预期。
    \item 前置递增 `++iterator` 和后置递增 `iterator++` 操作成功,能够正确遍历链表并访问元素。
    \item 初始化列表构造函数工作正常,链表元素按照给定顺序插入。
    \item 拷贝构造和移动构造测试通过,链表的深拷贝和资源转移行为正常。
    \item `insert()` 函数成功在链表的头部和尾部插入元素,`erase()` 函数能够删除指定的范围。
    \item `clear()` 函数能够正确清空链表,且调用 `empty()` 和 `size()` 结果符合预期。
    \item 最终验证了空链表下的迭代器行为,`begin()` 和 `end()` 的结果正确。
\end{itemize}

此外,我还使用了 `valgrind` 对程序进行内存泄漏测试,结果显示程序在整个运行过程中没有发生内存泄漏,说明链表的所有动态分配的内存都得到了正确的管理和释放。

\section{(可选)bug报告}

测试过程中没有发现功能上的问题或 bug,所有功能都符合预期。

\end{document}

