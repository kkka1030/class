\documentclass[a4paper,12pt]{article}
\usepackage{amsmath}
\usepackage{amssymb}
\usepackage{geometry}
\usepackage[UTF8]{ctex}
\geometry{a4paper, margin=1in}

\title{第四章总结:树}
\author{}
\date{}

\begin{document}

\maketitle

\section*{总结}

\subsection*{树的基本结构}
\begin{itemize}
    \item 树是一种递归定义的数据结构,用于表示具有层次关系的元素。
    \item 二叉树是树的基本形式,每个节点最多有两个子节点。
\end{itemize}

\subsection*{二叉搜索树(BST)}
\begin{itemize}
    \item 左子树的所有节点值小于根节点,右子树的所有节点值大于根节点。
    \item 常见操作:插入、删除、查找。
    \item 平均时间复杂度为 $O(\log n)$,最坏情况为 $O(n)$。
\end{itemize}

\subsection*{平衡树}
为避免 BST 退化成链表,提出了以下平衡树:
\begin{itemize}
    \item \textbf{AVL 树}:严格平衡的二叉搜索树,通过旋转操作保持平衡。
    \item \textbf{Splay 树}:访问节点时进行旋转,将节点移到树根,优化后续访问。
    \item \textbf{B 树}:适合外部存储的大型数据集管理。
\end{itemize}

\subsection*{树的遍历方式}
\begin{itemize}
    \item \textbf{前序遍历}:根节点 $\rightarrow$ 左子树 $\rightarrow$ 右子树。
    \item \textbf{中序遍历}:左子树 $\rightarrow$ 根节点 $\rightarrow$ 右子树。
    \item \textbf{后序遍历}:左子树 $\rightarrow$ 右子树 $\rightarrow$ 根节点。
    \item \textbf{层序遍历}:按层次逐层访问节点。
\end{itemize}

\subsection*{集合和映射}
\begin{itemize}
    \item C++ 提供了 \texttt{set} 和 \texttt{map} 容器,分别用于存储集合和键值对。
    \item 它们基于红黑树实现,操作复杂度为 $O(\log n)$。
\end{itemize}

\subsection*{应用场景}
\begin{itemize}
    \item 树广泛用于表达式解析、文件系统、数据库索引等领域。
    \item 平衡树(如 AVL 树和 B 树)适用于需要高效查询的大型数据集。
\end{itemize}

\section*{学习重点}
\begin{itemize}
    \item 理解树的基本结构和性质。
    \item 掌握二叉搜索树及平衡树的操作与应用场景。
    \item 熟悉树的遍历算法及其实现。
    \item 了解 \texttt{set} 和 \texttt{map} 容器的使用及其实现原理。
\end{itemize}

\end{document}
