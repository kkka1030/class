\documentclass[UTF8]{ctexart}
\usepackage{geometry, CJKutf8}
\geometry{margin=1.5cm, vmargin={0pt,1cm}}
\setlength{\topmargin}{-1cm}
\setlength{\paperheight}{29.7cm}
\setlength{\textheight}{25.3cm}
\usepackage{longtable}
% useful packages.
\usepackage{amsfonts}
\usepackage{amsmath}
\usepackage{amssymb}
\usepackage{amsthm}
\usepackage{enumerate}
\usepackage{graphicx}
\usepackage{multicol}
\usepackage{fancyhdr}
\usepackage{layout}
\usepackage{listings}
\usepackage{float, caption}

\lstset{
    basicstyle=\ttfamily, basewidth=0.5em
}

% some common command
\newcommand{\dif}{\mathrm{d}}
\newcommand{\avg}[1]{\left\langle #1 \right\rangle}
\newcommand{\difFrac}[2]{\frac{\dif #1}{\dif #2}}
\newcommand{\pdfFrac}[2]{\frac{\partial #1}{\partial #2}}
\newcommand{\OFL}{\mathrm{OFL}}
\newcommand{\UFL}{\mathrm{UFL}}
\newcommand{\fl}{\mathrm{fl}}
\newcommand{\op}{\odot}
\newcommand{\Eabs}{E_{\mathrm{abs}}}
\newcommand{\Erel}{E_{\mathrm{rel}}}

\begin{document}

\pagestyle{fancy}
\fancyhead{}
\lhead{邵奕涵, 3230103664}
\chead{数据结构与算法作业}
\rhead{Nov.30th, 2024}

\section{设计思路}
本项目旨在实现一个基于堆排序的算法,并与标准库提供的堆排序实现(\texttt{std::sort\_heap()})进行性能对比。整个项目分为以下几部分:
\begin{enumerate}
    \item \textbf{堆排序实现:}在\texttt{HeapSort.h}文件中实现了一个模板类,利用堆的性质对向量进行原地排序。
    \item \textbf{测试程序:}在\texttt{test.cpp}中编写了一个测试程序,生成不同类型的测试序列,分别应用自定义的堆排序和标准库排序,并记录时间与正确性。
\end{enumerate}

\section{关键函数及实现细节}
堆排序的实现主要依赖以下几个关键函数:
\begin{itemize}
    \item \textbf{\texttt{buildHeap}:} 构建最大堆,从最后一个非叶子节点开始依次下滤。
    \item \textbf{\texttt{percDown}:} 下滤操作,将节点与其子节点交换以维持堆的性质。
    \item \textbf{\texttt{sort}:} 堆排序主函数,先构建最大堆,然后依次将堆顶元素与堆尾交换,最后对堆的有效部分重新下滤。
\end{itemize}



\section{测试流程}
写了一个check函数检测排序的正确性。同时,测试程序生成以下四种长度为1,000,000的序列:
\begin{itemize}
    \item \textbf{随机序列:} 使用随机数生成器生成随机数。
    \item \textbf{有序序列:} 依次递增的整数序列。
    \item \textbf{逆序序列:} 依次递减的整数序列。
    \item \textbf{部分重复序列:} 元素为有限个值的重复序列。
\end{itemize}

每种序列分别使用自定义堆排序和\texttt{std::sort\_heap}排序,记录排序时间并验证排序正确性。



\section{测试结果}
以下是四种序列的测试结果,包含两种排序的时间对比。

\begin{longtable}{|l|c|c|c|}
\hline
\textbf{测试类型} & \textbf{HeapSort时间(ms)} & \textbf{std::sort\_heap时间(ms)} & \textbf{效率提升} \\ \hline
随机序列        & 339                      & 506                              & 快了33.0\%        \\ \hline
有序序列        & 249                      & 375                              & 快了33.6\%        \\ \hline
逆序序列        & 235                      & 395                              & 快了40.5\%        \\ \hline
部分重复序列    & 277                      & 422                              & 快了34.3\%        \\ \hline
\end{longtable}

\section{结论}
通过测试可以得出:
\begin{enumerate}
    \item 自定义的堆排序在所有测试序列中均能正确排序,性能优于标准库的\texttt{std::sort\_heap()}。
    \item 在逆序序列中,自定义堆排序的效率优势最为显著。
    \item 在部分重复序列中,自定义堆排序的时间也具有明显的性能提升。
    
\end{enumerate}

整体来看,自定义堆排序是一个高效、稳定的排序实现,适用于各种场景。
\section{时间复杂度分析}

\begin{enumerate}
    \item 构建最大堆的时间复杂度为O(N).
    \item 下滤操作的时间复杂度为O(logN).
    \item 于是堆排序的时间复杂度为O(N+NlogN)=O(NlogN)
    
\end{enumerate}
\end{document}

