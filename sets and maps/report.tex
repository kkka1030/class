\documentclass[a4paper,12pt]{article}
\usepackage{amsmath}
\usepackage{amssymb}
\usepackage{geometry}
\usepackage{listings}
\usepackage[UTF8]{ctex}
\geometry{a4paper, margin=1in}

\title{Sets 和 Maps 的介绍及示例}
\author{}
\date{}

\begin{document}

\maketitle

\section*{1. Sets 的特点和用法}

\subsection*{特点}
\begin{itemize}
    \item \texttt{set} 是一种有序容器,其中的元素按键值排序,并且不允许重复。
    \item 常见操作包括插入、删除和查找,时间复杂度为 $O(\log n)$。
    \item 提供迭代器 (\texttt{iterator, const\_iterator}) 用于遍历容器。
\end{itemize}

\subsection*{示例代码:基本使用}
\begin{lstlisting}[language=C++, caption=Set 示例]
#include <iostream>
#include <set>
using namespace std;

int main() {
    set<int> mySet;

    // 插入元素
    mySet.insert(10);
    mySet.insert(20);
    mySet.insert(10); // 重复元素被忽略

    // 遍历元素
    for (int val : mySet) {
        cout << val << " "; // 输出: 10 20
    }
    cout << endl;

    // 查找元素
    if (mySet.find(20) != mySet.end()) {
        cout << "20 存在于集合中" << endl;
    }

    // 删除元素
    mySet.erase(10);
    cout << "集合大小: " << mySet.size() << endl;

    return 0;
}
\end{lstlisting}

\subsection*{自定义排序规则}
通过比较函数自定义排序,例如降序:
\begin{lstlisting}[language=C++, caption=Set 自定义排序]
struct Compare {
    bool operator()(int a, int b) const {
        return a > b; // 降序
    }
};

set<int, Compare> mySet = {10, 20, 5};
for (int val : mySet) {
    cout << val << " "; // 输出: 20 10 5
}
\end{lstlisting}

\section*{2. Maps 的特点和用法}

\subsection*{特点}
\begin{itemize}
    \item \texttt{map} 是一种有序的键值对容器,键(\texttt{key})唯一,值(\texttt{value})可以重复。
    \item 支持键到值的高效查找,操作复杂度为 $O(\log n)$。
    \item 提供类似数组的索引操作符 \texttt{[]}。
\end{itemize}

\subsection*{示例代码:基本使用}
\begin{lstlisting}[language=C++, caption=Map 示例]
#include <iostream>
#include <map>
using namespace std;

int main() {
    map<string, int> myMap;

    // 插入键值对
    myMap["Alice"] = 30;
    myMap["Bob"] = 25;

    // 遍历键值对
    for (auto &entry : myMap) {
        cout << entry.first << ": " << entry.second << endl;
    }

    // 查找键
    if (myMap.find("Alice") != myMap.end()) {
        cout << "Alice 存在于映射中" << endl;
    }

    // 删除键
    myMap.erase("Bob");
    cout << "Alice 的年龄: " << myMap["Alice"] << endl;

    return 0;
}
\end{lstlisting}

\subsection*{自定义比较规则}
\begin{lstlisting}[language=C++, caption=Map 自定义排序]
struct Compare {
    bool operator()(const string &a, const string &b) const {
        return a > b; // 按键降序排列
    }
};

map<string, int, Compare> myMap;
myMap["Alice"] = 30;
myMap["Bob"] = 25;
for (auto &entry : myMap) {
    cout << entry.first << ": " << entry.second << endl; // 输出: Bob, Alice
}
\end{lstlisting}

\end{document}
